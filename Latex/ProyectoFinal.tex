\documentclass[12pt]{report}

%\usepackage{titleps}
\usepackage{fancyhdr}
\usepackage{graphicx}
\usepackage[spanish]{babel}
\usepackage[utf8]{inputenc}



\pagestyle{myheadings}
\pagestyle{fancy}
\fancyhf{}
\setlength{\headheight}{33pt}
\renewcommand{\headrulewidth}{2pt}
\renewcommand{\footrulewidth}{2pt}

\fancyhead[L]{\includegraphics[width=1cm]{LogoUTN.png}}
\fancyhead[C]{}
\fancyhead[R]{Universidad Tecnológica Nacional - Facultad regional Córdoba}
\fancyfoot[R]{\thepage}



\begin{document}

\begin{titlepage}

    \begin{center}
    \vspace*{-1in}
    \begin{figure}[htb]
    \begin{center}
    \includegraphics[width=8cm]{LogoUTN.png}
    \end{center}
    \end{figure}
    
    FACULTAD REGIONAL CÓRDOBA\\
    \vspace*{0.15in}
    DEPARTAMENTO DE INGENIERÍA ELECTRÓNICA \\
    \vspace*{0.6in}
    \begin{large}
    PROYECTO FINAL:\\
    \end{large}
    \vspace*{0.2in}
    \begin{Large}
    \textbf{RED MULTINODAL PARA DETECTAR INHIBICIONES EN SISTEMAS DE SEGURIDAD VEHICULAR} \\
    \end{Large}
    \vspace*{0.3in}
    \begin{large}
    Coronel Martín, Fantin Stéfano, Giletta Julian\\
    \end{large}
    \vspace*{0.3in}
    \rule{80mm}{0.1mm}\\
    \vspace*{0.1in}
    \begin{large}
    Docentes evaluadores: \\
    Candiani, Carlos\\
    Rabinovich, Daniel\\
    Galleguillo, Juan\\
    \end{large}
    \end{center}
    
\end{titlepage}


\chapter*{}
\pagenumbering{Roman} % para comenzar la numeracion de paginas en numeros romanos
\begin{flushright}
\textit{Agradecemos profundamente a nuestra familia \\
que siempre nos apoyó en este camino \\
y a la Universidad Tecnológica Nacional, \\
particularmente a la carrera de ingeniería electrónica,
la cual siempre se caracterizó por la buena organización y la búsqueda del bienestar estudiantil.}
\end{flushright}
\tableofcontents


\pagenumbering{arabic}
\section{Capitulo}
\subsection{One}
\pagebreak
\subsection{Two}



\end{document}

% vamos a resoetar estas estructura: http://www.cyta.com.ar/biblioteca/bddoc/bdlibros/guia_tesis/guia_tesis_archivos/principal.htm 
% buenisima guia de latex aplicada a tesis http://minisconlatex.blogspot.com/2011/04/como-escribir-una-tesis-con-latex.html