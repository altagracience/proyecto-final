\documentclass[12pt]{report}

%\usepackage{titleps}
\usepackage{fancyhdr}
\usepackage{graphicx}
\usepackage[spanish]{babel}
\usepackage[utf8]{inputenc}
\usepackage{subfigure} 
\usepackage{todonotes}
\usepackage{hyperref}


\pagestyle{myheadings}
\pagestyle{fancy}
\fancyhf{}
\setlength{\headheight}{33pt}
\renewcommand{\headrulewidth}{2pt}
\renewcommand{\footrulewidth}{2pt}

\fancyhead[L]{\includegraphics[width=1cm]{LogoUTN.png}}
\fancyhead[C]{}
\fancyhead[R]{Universidad Tecnológica Nacional - Facultad regional Córdoba}
\fancyfoot[R]{\thepage}



\begin{document}


\begin{titlepage}

    \begin{center}
    \vspace*{-1in}
    \begin{figure}[htb]
    \begin{center}
    \includegraphics[width=8cm]{LogoUTN.png}
    \end{center}
    \end{figure}
    
    FACULTAD REGIONAL CÓRDOBA\\
    \vspace*{0.15in}
    DEPARTAMENTO DE INGENIERÍA ELECTRÓNICA \\
    \vspace*{0.6in}
    \begin{large}
    PROYECTO FINAL:\\
    \end{large}
    \vspace*{0.2in}
    \begin{Large}
    \textbf{RED MULTINODAL PARA DETECTAR INHIBICIONES EN SISTEMAS DE SEGURIDAD VEHICULAR} \\
    \end{Large}
    \vspace*{0.3in}
    \begin{large}
    Coronel Martín, Fantin Stéfano, Giletta Julian\\
    \end{large}
    \vspace*{0.3in}
    \rule{80mm}{0.1mm}\\
    \vspace*{0.1in}
    \begin{large}
    Docentes evaluadores: \\
    Candiani, Carlos\\
    Rabinovich, Daniel\\
    Galleguillo, Juan\\
    \end{large}
    \end{center}
    
\end{titlepage}

\pagenumbering{roman}

\chapter*{}
\pagenumbering{Roman} % para comenzar la numeracion de paginas en numeros romanos
\begin{flushright}
\textit{Agradecemos profundamente a nuestra familia \\
que siempre nos apoyó en este largo camino \\
y a la Universidad Tecnológica Nacional, \\
particularmente a la carrera de ingeniería electrónica,
la cual siempre se caracterizó por la buena organización y la búsqueda del bienestar estudiantil.}
\end{flushright}

\chapter*{Resumen} % si no queremos que añada la palabra "Capitulo"
\addcontentsline{toc}{section}{Resumen} % si queremos que aparezca en el índice
\markboth{RESUMEN}{RESUMEN} % encabezado

En este documento se plasma el proceso de investigación y desarrollo de un sistema multinodal pensado para detectar 
inhibiciones en los sistemas de seguridad vehicular que funcionen en la frecuencia de 433,92MHz.\par
El dispositivo planteado cuenta con tres unidades de recepción, las cuales denominamos nodos, y una central de procesamiento
encargada de comunicarse y gestionar la información por estos recolectada. \par
Para la comunicación entre los nodos y la central se utiliza el protocolo RS485 
y para comunicar la central con un servidor web, teniendo así los datos a disposición remotamente, se hace uso de un módulo GSM.


\tableofcontents % indice de contenidos

\cleardoublepage
% \addcontentsline{toc}{chapter}{Lista de figuras} % para que aparezca en el indice de contenidos
\listoffigures % indice de figuras

\cleardoublepage
% \addcontentsline{toc}{chapter}{Lista de tablas} % para que aparezca en el indice de contenidos
% \listoftables % indice de tablas

\pagenumbering{arabic}
\chapter{Introducción}
Hoy en día en muchos paises, y particularmente en la Argentina, se presenta una recurrente modalidad de delincuencia que trata de 
inhibir los sistemas de seguridad vehicular, no permitiendo que estos se cierren y pudiendo tener completo acceso a su interior. Es 
una metodología muy usada debido a que no se hace uso de la fuerza bruta para ingresar al vehículo y apela a la distracción del usuario.\par
Siendo conscientes de esta problemática nos hemos empeñado en desarrollar un sistema de detección de los dispositivos utilizados
con este fin. Como se verá más adelante se ha hecho un relevamiento de los dispositivos incautados por la policía a través de notas
periodísticas y con vínculos internos a departamentos policiales que pusieron a disposición la información presente sobre estos.\par
Los inhibidores pueden operar corrompiendo la trama de datos emitida por el llavero, no dejando así que el receptor del vehículo 
pueda identificar el intento de comunicación y también lo pueden hacer saturando el receptor. Creemos importante que el dispositivo a diseñar 
abarque estas dos posibilidades. \par
Otra característica importante a la hora de encarar el proyecto es determinar la frecuencia de operación. Los controles remotos poseen transmisores
de radio de corto alcance que operan en dos bandas posibles: 433,92 MHz para vehículos de origen europeo y asiático y 315 MHz para vehículos de origen 
norteamericano. En la Argentina la mayor cantidad de sistemas de seguridad operan en 433,92 MHz por lo que nos pareció adecuado diseñar el
detector para esta frecuencia. \par
Una vez definidos los requerimientos básicos del desarrollo es importante establecer el lugar en el que creemos adecuado que opere. Es así
que surge la idea de tener al menos tres nodos receptores capaces de identificar si hay o no un inhibidor en las inmediaciones de este
y que la información que recolecte sea enviada a una unidad de procesamiento, que denominamos ''central'', la cual se encargaría de 
comunicarse con los nodos, recopilar la información y subirla a una base de datos, permitiendo la visualización remota de lo que está sucediendo 
en tiempo real y, de ser posible, triangular la posición estimada del dispositivo inhibidor dentro del arreglo de receptores.\par
Esto sería emplazado en un estacionamiento utilizando una estrategia de disposición que se analizará más adelante


\section{Marco teórico}


Es importante realizar un estudio profundo sobre el tema que vamos a abordar, ya que es necesario definir un método novedoso que satisfaga
la necesidad de distinguir señales legítimas generadas por un control remoto de interferencias.

\subsection{Codificación en sistemas de seguridad vehicular}

Desde los inicios de los sistemas remotos de apertura y control vehicular hasta ahora se ha transitado un largo camino. 
El primer sistema de identificación por radiofrecuencia fue ingresado en el mercado por Renaul en el modelo Fuego en el año 1995.
Todo este tiempo desde su puesta en uso hasta la fecha ha servido para definir y universalizar las metodologías usadas para comunicarse,
intentando dar una mejora en cuanto a la seguridad y efectividad del sistema.\par

\subsubsection{Sistemas de código fijo}

Esta es la forma más difundida de codificación para los controles remotos vehiculares en nuestro país. Se trata de un código de comunicación
fijo, que precisa estar preestablecido en el circuito integrado del dispositivo, el cual se mantiene constante para la acción a realizar.
De esto podemos notar que para los controles remotos comunes que poseen opción de cierre y apertura del automóvil se tienen solo dos códigos
fijos que realizan cada una de estas acciones y que, eventualmente, podrían ser copiados y replicados para generar la acción codificada. 

\subsubsection{Sistemas de código variable}

Esta metodología no está muy difundida en nuestra región. Se trata de un sistema de seguridad que no repite el mismo patrón para ejecutar la 
acción de cierre o apertura del vehículo para evitar que se pueda leer y replicar el código. Usualmente se hace uso de un generador de números 
pseudoaleatorios que se encuentra en el emisor y receptor, un contador de pulsaciones en el emisor y un contador de recepciones en el vehículo.
Cuando el control remoto envía la señal para realizar una acción en el vehículo este manda su contador, el cual será comparado con el 
interno del receptor y, de estar dentro de la ventana de aceptación definida en el sistema de seguridad, el automóvil autentica el mensaje 
recibido y actualiza el contador interno, ya que este puede diferir al de la llave.
Hay diversos tipos de encriptación de la comunicación; aquí solo mencionaremos los más difundidos: Hitag 1, Hitag 2, Hitag AES, DST-40, Keeloq

\subsubsection{Sistemas por desafío}

El sistema por desafío es actualmente el más utilizado en autos de alta gama. En este caso el control remoto intenta comunicarse y
el vehículo envía una pregunta desafío que tiene que ser respondida correctamente para validar la comunicación.\par
En esta variante se puede observar que es necesario que el control remoto y el vehículo tengan la capacidad de
emitir y recibire datos, generando una comunicación bidireccional.
Hay diversas opciones de desafíos de requerimiento realizados por el vehículo, pero la más utilizada es la de validación de contraseña, donde el desafío pedido
es pedir la contraseña y esta será o no validada. Esto en definitiva no impide que sea replicado el patrón de comienzo de comunicación y 
la autenticación, por lo que hay modalidades más avanzadas como tener una tabla de códigos pseudoaleatorios definida en ambos dispositivos
y asociada a un identificador, de modo que el vehículo requiera el código por medio de este no dando lugar a que un escucha externo pueda saber
a qué valor está asociado.

\subsection{Estructura de transmisión}

Tener noción previa de lo que esperamos recibir cuando hacemos un análisis de una señal es de gran importancia, por lo que en esta sección 
analizaremos la estructura de transmisión de un control remoto de autos.\par
Como antes fue mencionado no hay solo una frecuencia de operación, pero sí hay una que es ampliamente difundida en nuestro país y en esa nos 
centraremos (433,92 MHz), la modulación utilizada en la mayor cantidad de estos dispositivos es ASK, por su fácil implementación. Con esta
información ya seríamos capaces de demodular la señal y analizar la estructura.\par 
Para la demodulación de la señal hemos utilizado un SDR (Software Defined Radio) como el que se puede observar en la figura \ref{SDR}, el cual
fue facilitado por el centro de investigación G.In.T.E.A (Grupo de Investigación y Transferencia en Electrónica Avanzada) de la Universidad
Tecnológica Nacional, facultad regional Córdoba.\par

\begin{figure}[htb]
	\centering
	\includegraphics[scale=0.4]{sdr.png}
	\caption{Software Defined Radio utilizado para tomar las primeras mediciones}
	\label{SDR}
\end{figure}

En la figura \ref{llaves} podemos observar las primeras mediciones tomadas. Aquí podemos distinguir la estrategia de transmisión que se
utiliza. En un comienzo la señal posee un preámbulo, el cual es utilizado por el receptor para sincronizar el reloj del receptor  para 
decodificar correctamente los paquetes del transmisor. Después del preámbulo hay una palabra de sincronización que se utiliza para evitar 
choques con otros dispositivos que operan en esa banda y por último se encuentra la señal de código real.\par
Al presionar el botón del control remoto el preámbulo es enviado una única vez y luego se envía la palabra de sincronización y el
comando de acción repetidamente hasta que se deje de accionar. El espectro de la señal transmitida se puede observar en la figura \ref{espectro_ASK},
la cual es una simulación en el software AWR de una transmisión de datos random modulados ASK.

\begin{figure}[htb]
	\centering
	\includegraphics[scale=0.4]{llaves.png}
	\caption{Demodulación ASK de señales de controles remotos en 433,92 MHz}
	\label{llaves}
\end{figure}

\begin{figure}[htb]
	\centering
	\includegraphics[scale=0.4]{espectro_ASK.png}
	\caption{Espectro de señal random bits modulada ASK en 433,92 MHz }
	\label{espectro_ASK}
\end{figure}

\subsection{Tipos de inhibiciones}

Un inhibidor, o en inglés jammer, es un dispositivo desarrollado con el objetivo de deteriorar la comunicación en un enlace de 
radiofrecuencia. Esto puede ser logrado mediante dos estrategias:

\begin{itemize}
    \item Inhibición por corrupción de datos
    \item Inhibición por saturación de etapa receptora

\end{itemize}

\subsubsection{Inhibición por corrupción de datos}

El ataque más evidente que se presenta para inhibir una comunicación es el de inyectar en el canal que se desea perjudicar una señal con 
datos aleatorios que perjudique la relación señal ruido (SNR) y dificulte la recepción para el sistema. \par
En el caso particular de los vehículos, los receptores de radiofrecuencia que se utilizan y sobre los que basamos nuestro análisis
son de 433,92 MHz con un filtro de ancho de banda de entrada de 300 KHz -como se analiza en [1].\par
El ancho de banda de recepción da lugar a sumar ruido en el canal, alterando así los datos recibidos por el demodulador. Una figura
ilustrativa se puede observa en la imagen \ref{fpb_jam} de [2].

\begin{figure}[htb]
	\centering
	\includegraphics[scale=0.8]{fpb_jam.png}
	\caption{Presencia de inhibidor en ancho de banda de recepción}
	\label{fpb_jam}
\end{figure}

Existen diversas alternativas para efectivizar este tipo de interferencias. En la figura \ref{fpb_jam} se observa que se ha inyectado una
interferencia de ancho de banda angosto, pero también podría sumarse un tono, multitonos o sumar una señal de gran ancho de banda que 
tape completamente el canal. \par
Las alternativas antes mencionadas hacen referencia a inhibidores no inteligentes, los cuales están metiendo ruido constantemente. Hay otras 
alternativas de inhibiciones que de manera continua están escuchando el canal y cuando detectan una señal que 
desean interferir comienzan a emitir el ruido. Estos casos serán detallados más adelante.

\subsubsection{Inhibición por saturación de etapa receptora}

Los receptores de radiofrecuencia usualmente están diseñados asumiendo que se recibirá una pequeña señal de entrada, por lo que la primer
etapa presente es un amplificador de bajo ruido.  Este es clave para que el ruido del mezclador no afecte la relación señal ruido de las 
etapas siguientes. Entre las especificaciones importantes de dichos amplificadores de RF se incluyen la figura de ruido, la ganancia y la 
intercepción de intermodulación de tercer orden.\par \todo[inline, inlinewidth=4cm, noinlinepar]{[6]-[7]}
La influencia de grandes señales de interferencia se manifiesta de varias formas. Una de estas es en la intermodulación de tercer orden en 
la que dos señales, una pequeña (de interés) y la interferente (de gran amplitud), se superponen. La interferente podría saturar el receptor
de modo que la señal de interés presente una pequeña ganancia. Este efecto es causado por la no linealidad de tercer orden del sistema.

La saturación de un sistema suele tener un comportamiento de compresión  de la ganancia, decrementando la misma a medida de que la entrada aumenta.
Este efecto puede ser cualificado como el punto de 1 dB de compresión el cual está definido como el punto en el que la amplitud de la señal de 
entrada genera que la ganancia caiga 1 dB. Esto antes mencionado está claramente ilustrado en la figura \ref{eq:tercer_orden}.

\begin{equation}\label{eq:tercer_orden}
    y(t) \approx  a_1 x(t) + a_2 x^{2}(t) + a_3 x^{3}(t) 
\end{equation}

Donde \emph{y} es la salida del sistema y \(a_1, a_2, a_3 \) son coeficientes. Ahora supongamos que la entrada, como es de esperar con lo antes
descripto, resulta:

\begin{equation}\label{eq:entrada}
    x(t)=V_1 cos(\omega_1 t)+ V_2 cos(\omega_2 t)
\end{equation}

\(V_1\) representando a la señal de interés y \(V_2\) a la interferente.\par
Reemplazando en la ecuación \ref{eq:entrada} en \ref{eq:tercer_orden} y asumiendo que la interferencia es mucho más grande que la señal, 
la salida del sistema en la frecuencia de interés \(\omega_1\) resulta ser \ref{eq:intermodulacion}.

\begin{equation}\label{eq:intermodulacion}
    y(t) \approx \left( a_1 x(t) + \frac{3}{2} a_3 V_2^{2} \right) V_1  cos(\omega_1 t)
\end{equation}

Para que el sistema comprima la ganancia, como es evidente que sucede, el producto \(a_1 a_3 < 0\). De aquí se puede observar entonces que
la salida del sistema en la frecuencia deseada es función de \(V_2^{2}\), que la ganancia decae saturando el sistema y por ende se decrementa 
la SNR. Esto es fácilmente observable en la figura \ref{no_linealidad_e_tercer_orden}.

\begin{figure}[htb]
	\centering
	\includegraphics[scale=0.4]{no_linealidad_e_tercer_orden.png}
	\caption{Compresión de la ganancia por no linealidad de tercer orden}
	\label{no_linealidad_e_tercer_orden}
\end{figure}

\subsection{Estrategias de inhibición}

Antes hemos presentado los principios de inhibición que se pueden utilizar, en este apartado se tratará de una manera generalizada las diversas 
estrategias existentes para inhibir sistemas de comunicación. Cada una de estas estrategias tienen ventajas y desventajas, por lo que es necesario
hacer un análisis del ámbito de aplicación para elegir la más adecuada.


\subsubsection{Inhibición por ruido de banda ancha}

La característica principal de este tipo de estrategia es que introduce energía dentro de todo el ancho del espectro donde opera la comunicación.
Es aplicable a cualquier tipo de señal y es ideal para inhibir comunicaciones que tienen destinada una gran parte del espectro de frecuencias.\par
Este método tiene una fuerte desventaja y es que la potencia de interferencia aportada en el canal deseado tiene una muy baja densidad debido 
a es aplicada a un gran ancho de banda.

\subsubsection{Inhibición por ruido de banda parcial}

Este método opera inyectando ruido en bandas específicas del espectro, de modo que se efectúa la  inhibición en zonas de interés. Estas zonas 
pueden ser continuas o discontinuas, por lo que se destina más inteligentemente la potencia consumida. El ejemplo más trivial aplicado a nuestro
campo sería el de un inhibidor que funcione en 433,92 y en 315 MHz, siendo  aplicable a todos los canales de comunicación de controles remotos.

\subsubsection{Inhibición por ruido de banda angosta}

Esta caso es el más utilizado en el campo de inhibición sobre el que nos centramos ya que permite puntualizar la potencia en una pequeña
banda aumentando la densidad de potencia espectral. Para su aplicación es necesario conocer precisamente el canal a atacar debido a que las 
comunicaciones innalámbricas de banda angosta poseen un angosto filtrado.

\subsubsection{Inhibición por tono}

Se utiliza una señal constante que se modula con la portadora resultando una señal de muy angosto ancho de banda. En sistemas de comunicación 
avanzados posee una alta eficiencia de interferencia ya que perjudica la recuperación de la sintonización a causa de que el receptor detecta 
la señal como una segunda portadora y es la que más potencia por Hertz posee (densidad de potencia espectral) gracias a que está más concentrado
en el canal, como profundamente se analiza en [8].

\subsubsection{Inhibición por pulsos}

En esta estrategia nos enfocamos en el tiempo que se genera la interferencia. Se hace uso de uno de los métodos anteriores de inhibición
y se desata la misma de manera inteligente. De aquí surge el concepto de aplicar la estrategia a casos específicos, permitiendo, por ejemplo:
romper tramas específicas de datos conocidas cuando se detecta un CLT/RTS, interferir el dato de direccionamiento MAC o romper exclusivamente
la trama de datos.

\subsubsection{Inhibición por barrido y seguimiento}

Esta es una aplicación del ruido de banda parcial. Se realiza una variación rápida del posicionamiento espectral de la interferencia para inhibir
un gran ancho de banda teniendo un mejor aprovechamiento de la potencia disponible.\par
De esta alternativa se desprende la capacidad de seguimiento de la señal inhibidora, permitiendo contrarestar estrategias de comunicación que 
hacen uso de saltos de canales para ser efectivas.

\begin{figure}[htb]
	\centering
	\includegraphics[scale=0.65]{tipos_de_jam.png}
    \caption{Donde a) es el canal a inhibir, b) inhibición por ruido de banda ancha, c) inhibición por ruido de banda parcial continuo, 
    d) inhibición por ruido de banda parcial discontinuo, e) inhibición por ruido de banda angosta, f) inhibición por tono }
	\label{tipos_de_jam}
\end{figure}

\subsubsection{Simulación de inhibición por ruido de banda parcial}

En este apartado se ha elegido la metodología de inyección de ruido de banda parcial para ser simulado y mostrar como decae la calidad de 
comunicación y, por ende, la capacidad de recepción de la información. La simulación fue realizada con el software AWR de Cadence.\par
El diagrama en bloques se puede observar en la figura \ref{bloques_inh}, este cuenta de 4 secciones principales:

\begin{itemize}
    \item Inhibidor de potencia variable: este bloque inyecta ruido blanco al sistema en la frecuencia definida como portadora, que en nuestro 
    caso es 433,92 MHz. La potencia de ruido va a variar entre 0 y -30 dBW, lo que sería igual a decir entre -30 y -60 dBm. 
    \item Emisor de señal: el emisor de señal es un modulador de ASK que modula una generación aleatoria de bits a 2500 baudios con la portadora.
    \item Medio de enlace: en este caso está representado con un combinador de señal RF, el cual va a servir para sumar la potencia de las dos
    señales anteriores.
    \item Demodulador: en esta instancia se produce la demodulación de la señal ASK más el ruido blanco agregado. Al final posee un bloque que
    se encarga de controlar el BER (Bit Error Rate), chequeando cuántos datos de los enviados efectivamente fueron recibidos. 

\end{itemize}

\begin{figure}[htb]
	\centering
	\includegraphics[scale=0.37]{bloques_inh.png}
    \caption{Diagrama en bloques de sistema de comunicación ASK inhibido}
	\label{bloques_inh}
\end{figure}

Como antes se menciona, y como se puede observar en la figura \ref{BER_ask}, el error de recepción alcanza su valor máximo cuando el ruido 
inyectado es de 0 dBW (-30 dBm). En la figura se puede leer un valor de BER = 0.4866, el cual es lógico debido a que la modulación ASK solo 
posee dos símbolos, por lo que la probabilidad de que coincida el dato generado por el ruido y el esperado es del 50\%. Por otro lado el valor
mínimo de error en la simulación propuesta sucede cuando la potencia del inhibidor es de -30 dBW (-60 dBm) teniendo un error de cuatro bits 
por cada diez mil recibidos.

\begin{figure}[htb]
	\centering
	\includegraphics[scale=0.37]{BER_ask.png}
    \caption{Bit Error Rate para comunicación ASK inhibida.}
	\label{BER_ask}
\end{figure}

\subsection{Detección de inhibiciones}

En esta sección trataremos los métodos que pueden utilizarse para detectar interferencias en enlaces de radiofrecuencia. Es importante que 
sea robusta la detección, principalmente que el funcionamiento del sistema que se pretende asegurar no pueda desatar una falsa alarma. En el caso
puntual de aplicación de este proyecto el sistema debería poder identificar una inhibición y no debería reaccionar a las señales de controles 
remotos que van a estar funcionando en el área circundante. 
Hay algunas características las cuales son naturales pensar como sensibles de analizar para detectar inhibiciones, estas son: 

\begin{itemize}
    \item Potencia de la señal recibida.
    \item Sensado temporal de portadora.
    \item Ocupación del canal.

\end{itemize}

\subsubsection{Potencia de la señal recibida}

Como antes se ha definido, el receptor sufre compresión de ganancia en su primera etapa amplificadora cuando la potencia recibida es alta
comparado a la potencia de señal que se espera recibir y para el que fue diseñado. Es por esto que resulta natural analizar los niveles de
potencia recibidos, estableciendo un valor a partir del cual se señale como alarmante para la correcta recepción de la información. 
En el caso particular de los dispositivos de control remoto en los sitemas de seguridad vehicular resulta más sencillo el análisis 
debido a que los dispositivos emisores que se utilizan son de baja potencia comparado a la necesaria para saturar un receptor típico.\par
En [7] se realiza el análisis del receptor MAX1470 [1], muy difundido en sistemas de control remoto tanto en el ámbito automotriz como también
en sistemas de portones de apertura innalámbrica, sistemas de seguridad, sensores innalámbricos y mucho más. Este integrado es un receptor de 
ASK superheterodíneo de bajo costo que posee un mezclador de rechazo de imagen que mezcla la señal a una frecuencia intermedia de 10,7 MHz. En
la figura \ref{compresion_max} se puede observar que la ganancia del sistema de recepción se aplana aproximadamente con una entrada de -35 dBm.

\begin{figure}[htb]
	\centering
	\includegraphics[scale=0.5]{compresion_max.png}
    \caption{Curva de ganancia de MAX1470}
	\label{compresion_max}
\end{figure}


\subsubsection{Sensado temporal de portadora}

Esta es una estrategia muy utilizada en sistemas de comunicación en que se realiza un envío efectivo de paquetes cuando se detecta que el canal 
de transmisión está desocupado. Esta característica hace sensible al sistema de ser engañado dejando presente una señal portadora de información
que engañe a los dispositivos emisores de la red.\par
En nuestro caso no nos detendremos a hacer un análisis profundo de esta metodología ya que es ajena a nuestro potencial mecanismo de detección
debido a que, como se menciona en 3.1.2, el control remoto emite señal cuando es accionado un botón que este posee, haciendolo de manera continua
hasta que deje de ser apretado. Es fácil ver que esta característica de la comunicación hace inútil aplicar esta estrategia de detección

\subsubsection{Ocupación del canal}

El sistema de comunicación que nos basamos para desarrollar este trabajo tiene la particularidad de que la comunicación unilateral que sucede
entre los dispositivos de emisión (controles remotos) y recepción (vehículos) tienen una muy baja tasa de ocupación de canal. Esto se debe a 
la trama de datos empleada. La misma, como antes fue explicado, posee un período de sincronización que es una rápida variación de estados, para
que el receptor pueda engancharse en fase a la recepción, y luego envío de paquetes de datos separados por espacios vacíos de información. Esta
característica nos da una relación de bits en alto recibidos respecto a los medidos de un valor porcentual muy bajo. Es por esto que esta medición
en el canal, usandola estratégicamente, nos puede dar mucha información de lo que está sucediendo.


\chapter{Características generales del diseño}
\section{Objetivos globales del sistema}

En base a la información recolectada establecemos los requerimientos base del que se parte para el diseño del producto final. Creemos adecuado
realizar la separación de requerimientos en primarios y secundarios, ya que el sistema a desarrollar está enmarcado en el proyecto final de la 
carrera de grado de ingeniería electrónica donde, en conjunto con la cátedra, se intenta que los proyectos puedan culminarse en un plazo de 
tiempo lógico para la obtención del título, dando la posibilidad de seguir explayándose en el mismo a posterior. \par

De este modo, como objetivos primarios se establecen:

\begin{itemize}
    \item Identificar la presencia de señales con una potencia suficiente para inhibir la comunicación: como en el marco teórico se ha 
    estudiado en profundidad, la etapa amplificadora receptora sufre una compresión de la ganancia cuando en su entrada hay presente 
    una señal de alto nivel de potencia. Es por esto que se establece como requerimiento del sistema poder identificaruna señal que cumpla con
    estas características.
    \item Medir la ocupación del canal: Ha quedado claro que la trama de comunicación de las llaves remotas con los receptores de los automóviles 
    poseen características de espaciado entre paquetes de datos transmitidos, es por esto que el sensado de la ocupación del canal se vuelve 
    una medida crucial para determinar si hay o no presencia de interferencias.
    \item Activar alarmas sonoras y visuales en caso de estar en presencia de una inhibición: el sistema debe tener la capacidad de determinar 
    si hay presente una inhibición en su área de operación y, de ser así, debe disponer de métodos para dar alerta local de lo que está sucediendo.
    \item Disponer de comunicación a sistemas externos complementarios: un requisito del sistema es que posea la capacidad de enviar la información recolectada
    a un lugar remoto. Se prefiere la utilización de una metodología de comunicación inhalámbrica y ampliamente distribuída.
    \item Cargar los datos en una base de datos: es importante que la información de las inhibiciones detectadas sea subida a una base de datos
    que permita visualizar de manera remota y en tiempo real lo que está sucediendo con los sistemas de seguridad activos, brindando la posibilidad 
    de generar estadísticas y observar qué tipos de inhibidores están operando en la zona.
    \item Orientar el diseño del proyecto a la optimización de costos y recursos: es muy importante para el curso del trabajo que el diseño 
    se realice haciendo uso racional de los recursos disponibles, apuntando a la posibilidad de producir muchas unidades del sistema de seguridad
    y obtener ganancias.

\end{itemize}

Como objetivo secundario se define:

\begin{itemize}
    \item Estimar la procedencia de la interferencia: el único objetivo secundario del sistema es que tenga la capacidad, mediante el método 
    más conveniente, de determinar la posición estimada de la fuente de interferencia activa. Esto está planteado de esta manera debido a que 
    se desconoce la factibilidad de su realización en el marco del proyecto de fin de grado de ingeniería electrónica.
\end{itemize}

\section{Esquema de funcionamiento}

En la figura \ref{bloques_funcionamiento} se puede observar el sistema planteado para solucionar el desafío de detectar inhibiciones en los sitemas
de seguridad vehicular. El mismo contará con nodos detectores que operarán en 433,92 MHz, un canal de comunicación hacia la central, haciendo
uso de un protocolo que más adelante se detallará y una central de operación donde se decidirá si hay o no inhibición en su área de operación
desatando las alarmas pertinentes y subiendo la información al servidor. \par

\begin{figure}[htb]
	\centering
	\includegraphics[scale=0.53]{bloques_funcionamiento.png}
    \caption{Diagrama en bloques sobre el funcionamiento del sistema global}
	\label{bloques_funcionamiento}
\end{figure}

\section{Subsistemas que lo componen}

En esta sección únicamente se hará mención de los subsistemas que componen al producto final, más adelante se detallará en capítulos el funcionamiento 
particular de cada uno de estos bloques.

\begin{itemize}
    \item Nodo receptor 
    \begin{itemize}
          \item Microcontrolador.
          \item Receptor de RF. 
          \item Integrado para comunicación entre nodos y central.
    \end{itemize}
    \item Central de procesamiento 
    \begin{itemize}
          \item Microcontrolador.
          \item Integrado para comunicación a la base de datos. 
          \item Integrado para comunicación entre nodos y central.
    \end{itemize}
    
\end{itemize}




\pagebreak




\end{document}

% vamos a resoetar estas estructura: http://www.cyta.com.ar/biblioteca/bddoc/bdlibros/guia_tesis/guia_tesis_archivos/principal.htm 
% buenisima guia de latex aplicada a tesis http://minisconlatex.blogspot.com/2011/04/como-escribir-una-tesis-con-latex.html