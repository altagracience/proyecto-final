\documentclass[12pt]{report}

%\usepackage{titleps}
\usepackage{fancyhdr}
\usepackage{graphicx}
\usepackage[spanish]{babel}
\usepackage[utf8]{inputenc}



\pagestyle{myheadings}
\pagestyle{fancy}
\fancyhf{}
\setlength{\headheight}{33pt}
\renewcommand{\headrulewidth}{2pt}
\renewcommand{\footrulewidth}{2pt}

\fancyhead[L]{\includegraphics[width=1cm]{LogoUTN.png}}
\fancyhead[C]{}
\fancyhead[R]{Universidad Tecnológica Nacional - Facultad regional Córdoba}
\fancyfoot[R]{\thepage}



\begin{document}

\begin{titlepage}

    \begin{center}
    \vspace*{-1in}
    \begin{figure}[htb]
    \begin{center}
    \includegraphics[width=8cm]{LogoUTN.png}
    \end{center}
    \end{figure}
    
    FACULTAD REGIONAL CÓRDOBA\\
    \vspace*{0.15in}
    DEPARTAMENTO DE INGENIERÍA ELECTRÓNICA \\
    \vspace*{0.6in}
    \begin{large}
    PROYECTO FINAL:\\
    \end{large}
    \vspace*{0.2in}
    \begin{Large}
    \textbf{RED MULTINODAL PARA DETECTAR INHIBICIONES EN SISTEMAS DE SEGURIDAD VEHICULAR} \\
    \end{Large}
    \vspace*{0.3in}
    \begin{large}
    Coronel Martín, Fantin Stéfano, Giletta Julian\\
    \end{large}
    \vspace*{0.3in}
    \rule{80mm}{0.1mm}\\
    \vspace*{0.1in}
    \begin{large}
    Docentes evaluadores: \\
    Candiani, Carlos\\
    Rabinovich, Daniel\\
    Galleguillo, Juan\\
    \end{large}
    \end{center}
    
\end{titlepage}

\pagenumbering{roman}

\chapter*{}
\pagenumbering{Roman} % para comenzar la numeracion de paginas en numeros romanos
\begin{flushright}
\textit{Agradecemos profundamente a nuestra familia \\
que siempre nos apoyó en este largo camino \\
y a la Universidad Tecnológica Nacional, \\
particularmente a la carrera de ingeniería electrónica,
la cual siempre se caracterizó por la buena organización y la búsqueda del bienestar estudiantil.}
\end{flushright}

\chapter*{Resumen} % si no queremos que añada la palabra "Capitulo"
\addcontentsline{toc}{section}{Resumen} % si queremos que aparezca en el índice
\markboth{RESUMEN}{RESUMEN} % encabezado

En este documento se plasma el proceso de investigación y desarrollo de un sistema multinodal pensado para detectar 
inhibiciones en los sistemas de seguridad vehicular que funcionen en la frecuencia de 433,92MHz.\\
El dispositivo planteado cuenta con tres unidades de recepción, las cuales denominamos nodos, y una central de procesamiento
encargada de comunicarse y gestionar la información por estos recolectada. \\
Para la comunicación entre los nodos y la central se utiliza el protocolo RS485, 
y para comunicar la central con un servidor web, teniendo así los datos a disposición remotamente, se hace uso de un módulo GSM.


\tableofcontents % indice de contenidos

\cleardoublepage
% \addcontentsline{toc}{chapter}{Lista de figuras} % para que aparezca en el indice de contenidos
\listoffigures % indice de figuras

\cleardoublepage
% \addcontentsline{toc}{chapter}{Lista de tablas} % para que aparezca en el indice de contenidos
\listoftables % indice de tablas

\pagenumbering{arabic}
\chapter{Introducción}
Hoy en día en muchos paises, y particularmente en la Argentina, se presenta una recurrente modalida de delincuencia que trata de 
inhibir los sistemas de seguridad vehicular, no permitiendo que estos se cierren y pudiendo tener completo acceso a su interior. Es 
una metodología muy usada debido a que no se hace uso de la fuerza bruta para ingresar al vehículo y apela a la distracción del usuario.\\
Siendo conscientes de esta problemática nos hemos empeñado en desarrollar un sistema de detección de los dispositivos utilizados
con este fin. Como se verá más adelante se ha hecho un relevamiento de los dispositivos incautados por la policía a través de notas
periodísticas y con vínculos internos a departamentos policiales que pusieron a disposición la información presente sobre estos.\\
Los inhibidores pueden operar corrompiendo la trama de datos emitida por el llavero, no dejando así que el receptor del vehículo 
pueda identificar el intento de comunicación y también lo pueden hacer saturando el receptor, cosa que de igual manera este no puede identificar
la comunicación intentada. Creemos importante que el dispositivo a diseñar abarque estas dos posibilidades. \\
Otra característica importante a la hora de encarar el proyecto es determinar la frecuencia de operación. Los controles remotos poseen transmisores
de radio de corto alcance que operan en dos bandas posibles: 433,92 MHz para vehículos de origen europeo y asiático y 315 MHz para vehículos de origen 
norteamericano. En la Argentina la mayor cantidad de sistemas de seguridad operan en 433,92 MHz por lo que nos pareció adecuado diseñar el
detector para esta frecuencia. \\
Una vez definidas los requerimientos básicos del desarrollo es importante establecer el lugar en el que creemos adecuado que opere. Es así
que surge la idea de tener al menos tres nodos receptores capaces de identificar si hay o no un inhibidor en las inmediaciones de este
y que la información que recolecte sea enviada a una unidad de procesamiento, que denominamos "central", la cual se encargaría de 
comunicarse con los nodos, recopilar la información y subirla a una base de datos, permitiendo la visualización remota de lo que está sucediendo 
en tiempo real y, de ser posible, triangular la posición estimada del dispositivo inhibidor dentro del arreglo de receptores.\\
Esto sería emplazado en un estacionamiento utilizando una estrategia de disposición que se analizará más adelante

\section{Objetivos de la investigación}
En base a la información recolectada hemos definido las bases de funcionamiento
capacidad de detectar inhibicion de potencia o corrupcion
\section{Estado del arte}
\pagebreak




\end{document}

% vamos a resoetar estas estructura: http://www.cyta.com.ar/biblioteca/bddoc/bdlibros/guia_tesis/guia_tesis_archivos/principal.htm 
% buenisima guia de latex aplicada a tesis http://minisconlatex.blogspot.com/2011/04/como-escribir-una-tesis-con-latex.html