\chapter{Conclusiones y trabajo futuro} \par

De acuerdo al trabajo previamente expuesto en este capítulo haremos mención de las conclusiones que pudimos extraer al 
finalizarlo y se evaluarán posibilidades de desarrollos futuros.  \par

\section{Conlcusiones} \par 

En primera instancia nos parece importante dar nuestra perspectiva respecto a la viabilidad del proyecto. El mismo fue
enfocado con carácter de costos minimizados buscando que el uso de los recursos monetarios disponibles sea el más eficiente
posible. De esta manera nos encontramos con un producto finalizado que luce muy adecuado para la fabricación en cantidad aunque 
se reconoce que la utilización de módulos prefabricados -como previamente se analiza- es muy beneficiosa para la producción
de los prototipos y primeros sistemas, pero en la posibilidad de producir en serie debería hacerse una adaptación de esto a
un sistema que integre todo los bloques en fabricación propia. En las primeras instancias de fabricación se observa que
muchos de los componentes utilizados precisan ser importados y, debido a la situación actual del país y a la alta carga impositiva
que esto implica, resulta no ser conveniente la compra de un pequeño número de dispositivos en el exterior, quedando así 
avalado el uso de algunos bloques componentes. \par 
Desde el punto de vista económico el sistema planteado cuenta con un punto débil el cual está definido en los requerimientos del
mismo. Se busca que el sistema de seguridad no pueda ser inhibido por un agente externo por lo que los nodos y 
la central se comunican entre sí de manera cableada. Resulta ser que el cableado debe ser de calidad que asegure la comunicación
RS485 y en el mismo la alimentación para los nodos. Es por esto que en la aplicación del sistema planteado la mayor 
cantidad de gastos reside en las tiradas de cable necesarias para emplazar el sistema en el lugar de operación. Esto podría
solucionarse con una metodología de comunicación inalámbrica pero daría lugar a altas vulnerabilidades. \par
La seguridad en la detección de inhibiciones en sistemas de seguridad de alarma de auto fue el eje central del desarrollo, por 
lo que siempre se buscó eliminar las vulnerabilidades que pudieran ocurrir. Es por esto que al momento de presentar el sistema
podemos decir que posee un sistema de detección robusto. Las pruebas de campo han sido muy variadas y han buscado eliminar
cualquier falla en el funcionamiento, ahora restaría el realizar un análisis en largo plazo de operación con una realimentación
del cliente que fuera a utilizarlo. \par
Como cierre, podemos decir que se ha consegudio un sistema que es capaz de detectar en fracciones de segundos señales que 
alteren el funcionamiento de seguridad inhalámbricos en la frecuencia de 433,92 MHz. El área de operación segura para que las
señales de baja potencia puedan ser trianguladas se estima que es de 50m a la redonda desde el punto central de una disposición
en forma triangular de los tres nodos, de igual modo esto podría ser ampliado con la penalización de que no todos los nodos en 
simultáneo alcancen a medir un valor de intensidad de señal en una inhibición por corrupción de datos. \par


\section{Trabajo futuro} \par

Después de terminar el trabajo enmarcado en el proyecto final de grado de ingeniería electrónica hemos podido divisar algunos
puntos sobre los cuales nos parece importante realizar mejoras en un futuro:

\begin{itemize}
    \item Frecuencia de operación: nuestro sistema fue diseñado para una única frecuencia de operación; como previamente fue
    expuesto existen dos principales en las que funcionan los receptores vehiculares, por lo que a futuro creemos importante
    que el sistema tenga la capacidad de detectar en ambas las inhibiciones presentes.
    \item Las inhibiciones y las estrategias de inhibición son diferentes para cada sistema de comunicación, por lo que creemos 
    interesante evaluar la posibilidad de detectar inhibiciones en diversos sistemas, aplicando e invetigando
    sobre métodos para determinar las inhibiciones particularmente para cada comunicación.
    \item El sistema de detección fue elaborado con tres nodos distribuidos espacialmente para tener la capacidad de predecir la
    procedencia de la señal. En nuestros planes cabe la posibilidad de desarrollar un sistema único portable que
    detecte inhibiciones a su alrededor, dando lugar a la venta de un producto para particulares. Este debería disponer de
    alarmas locales y un sistema de memoria de datos recolectados que puedan ser descargados por el usuario.
    
    \item Como antes fue mencionado es importante que en un futuro el diseño del sistema sea completamente integrado a una
    placa única, esto nos daría la posibilidad de reducir los tamaños y tener un sistema más redituable para ventas en cantidad. 

\end{itemize}

