\chapter{Ensayos}

En este capítulo hablaremos sobre los ensayos que se han realizado a lo largo del desarrollo del proyecto. Se llevaron a cabo diversas pruebas
sobre las diferentes partes que componen el sistema, en cada sección de este capítulo se exponen los ensayos realizados a cada una de esas partes
o subsistemas.

\section {Anális de llaves e inhibidores} 

El primer conjunto de ensayos que realizamos fueron enfocados a conocer la naturaleza de las comunicaciones de los sistemas de seguridad vehicular
y de los inhibidores. En esta instancia fue muy importante el uso de un SDR (Radio definida por software) para analizar el espectro de las señales y
llevar a cabo la demodulación, como ya explicamos en la sección 1.1.2. \par 
Las primeras mediciones, las realizamos a distintas llaves, como podemos observar en la figura \ref{llaves}. Los datos obtenidos mediante el SDR
fueron de gran importancia para conocer en profundiad la estructura de la comunicación de las llaves, información indispensable para establecer 
métodos de inhibición adecuados.
Como ya lo mencionamos en 1.1.3, hay dos tipos de inhibiciones, por lo cual fue necesario obtener al menos dos inhibidores que ataquen al sistema 
de seguridad de esas dos maneras. \par
El primer inhibidor que usamos, consiste sencillamente en un handie (o transceptor de radio portátil) que es capaz
de emitir en la misma frecuencia que las llaves de los autos. Debido a la potencia que posee logra saturar la etapa receptora de los automóviles.
Para lograr inhibir por corrupción de datos, fabricamos nuestro propio inhibidor, el cual emite un pulso de baja potencia dentro del ancho de banda
del receptor, logrando corromper la trama de comunicación.\par 

Los ensayos realizados con los inhibidores consistieron en conocer la vulnerbilidad de los sistemas de seguridad de los vehículos así como las
distancias y potencias que se requerían para inhibir. En el caso del handie, es capaz de lograr interferir en la comuncaicpon de la llave y el auto a
distancias grandes, debido a su alta potencia, mientras que en el caso del inhibidor por corrupción de datos, necesitaba situarce a distancias
cortas del vehículo. Esta información es muy importante para plantear un sistema realista y desarrollar una adecuada estrategia de detección.

\section {Pruebas y configuración del CC1101}

Una vez analizadas las características de las señales de interés con ayuda de un SDR, estábamos listos para probar el integrado elegido para la
recepción. En este ensayo utilizamos la Blue Pill para configurar los registros del CC1101 mediante comunicación SPI.
La primer prueba consistió simplemente en constatar que la comunicación SPI funcionada, esto se hizo configurando los registros con valores predeterminados 
y posteriormente requiriendo el valor de alguno de ellos desde la Blue Pill.\par 
Con la comunicación funcionando procedimos a encontrar los valores adecuados para los registros de configuración del CC1101.
Lo que se buscaba en este punto era configurar el integrado para que nos entregue la demodulación y el valor de RSSI de la señal que recibía, 
así como también dar con los valores adecuados de sensibilidad y ancho de banda. Para calcular los valores de los registros nos ayudamos del datasheet del CC1101
y del software SmartRF Studio, que recomienda el fabricante.

\section{Ensayos de las estrategias de detección}

En esta instancia, ya éramos capaces de extraer los datos relevantes de la señal recibida y accederlos mediante la Blue Pill, por lo que se procedió a
desarrollar un algoritmo que, mediante estos datos, pueda decidir si la señal recibida se trata de una interferencia emitida por un inhibidor.\par
En primer lugar se creo un mecanismo simple y, gracias a ensayos utilizando llaves y nuestros propios inhibidores, fuimos mejorando nuestras estrategias
de detección hasta volver al sistema confiable. La lógica de funcionamiento la podemos ver en las figuras \ref{estrategia_datos} y \ref{estrategia_rssi}\par 
Hasta ahora, la blue pill la utilizamos conectada a una computadora para poder debuggear los programas que ensayábamos y así encontrar fallos de forma
rápida. Sin embargo, en un producto final, debía ser capaz de actuar independiente, es por esto que en este punto desarrollamos el primer
prototipo de nodo, que ya mostramos en la figura \ref{prototipo_nodo}). Sobre este primer prototipo se realizaron diversos ensayos para terminar
de afinar la detección. \par

\section{Servidor Web y conexión a internet}

Paralelo al desarrollo del nodo, trabajamos en contruir un sitio web propio, capaz de recibir información y plasmarla en tablas y gráficos.
En una primer instancia, se ensayó el servidor web subiendo datos de forma manual, lo que nos permtía ver como oganizaba y mostraba los datos. \par
Una vez funcionando bien el sitio web, procedimos a realizar pruebas sobre el SIM800L. Recordemos que este módulo, es el responsable de dar a la central conexión
a internet mediante 2G.
El SIM800L se comunica mediante comandos AT, por lo que lo usamos en conjunto con la Blue Pill para relizar los ensayos. En primer lugar solamente
buscábamos que el sim800L logre conectarse a la red gracias a una SIM que le insertamos. Puede parece un paso simple, pero debido a que este módulo
posee requerimientos particulares en cuanto a la alimentación, fue importante realizar pruebas para desarrollar una fuente adecuada para el mismo.
Cuando logramos que el SIM800L se conecte con la red, ensayamos la capacidad que tenía de subir datos al servior mediante las órdenes recibidas desde 
la Blue Pill a través de los comandos AT.\par 

\section{Desarrollo de la comunicación serial del sistema}

En este punto ya logramos tener un nodo capaz de recibir señales y de detectar inhibiciones y un servidor web funcional al cual podemos subir datos
mediante el sim800l y una blue pill. El siguiente paso, era desarrollar una comunicación robusta entre nodos y la central. Fue en este momento que decidimos
usar el MAX485 para comunicar el sistema mediante RS485.\par
Al añadir el MAX485 al conjunto de sim800l y bluepill, queda conformado el primer prototipo en protoboard de nuestra central.\par
Los primeros ensayos consistieron simplemente en comunicar dos Blue Pill aisladas, esto era necesario para conocer el estandar de comunicación en cuestión
y familiarizarnos con el manejo de la UART del nuestro microcontrolador. El siguiente paso, fue añadir el prototipo del nodo con el agregado de un max485
y nuestro prototipo de central. En este punto todavía no estábamos enviando datos de recepción a través de la comunicación, solamente envíabamos datos
sin importancia para entender como conformar un red de rs485. Estos ensayos nos permitieron desarrollar una estrategia de comunicación confiable.\par
Fue en este momento, cuando desarrollamos nuestra trama de comunicación, y luego de varios ensayos definir por completo como se iban a comunicar los nodos
y la central. En este punto se realizaron numerosos ensayos del sistema funcionando en conjunto, desde la detección de la inhibición y la comunicación de los nodos
a la central, hasta encender alarmas y subir los datos a la web.
Cabe aclarar que todas las pruebas de comunicación realizadas hasta el momento se hicierona a cortas distancias con un par trensado telefónico.

\section{Ensayos Finales}

Al quedar ya conformado el formato de nuestro sistema, diseñamos nuestras placas definitivas y las mandamos a fabricar. Con nuestros nodos y la central,
llevamos a cabo los últimos ensayos, esta vez utilizando un cable de mayor longitud y situandonós en un espacio de gran tamaño.
Gracias a estas pruebas realizamos los últimos ajustes, para lograr un sistema robusto y funcional. \par
Por último se puso a pruebas el sistema en un estacionamiento real, obteniendo excelentes resultados.


